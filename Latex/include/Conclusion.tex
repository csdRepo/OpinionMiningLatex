% CREATED BY DAVID FRISK, 2015
\chapter{Conclusion}\label{ch_6}
In this Thesis, three self-contained mechanisms for data mining from users' texts that were published in the Greek Government's online service for Public consultation were presented. The information that is extracted by the three Mechanisms constitutes information of semantic nature. The three mechanisms for argument extraction, suggestions and overall opinion are to a great extent adapted for this particular corpus that we studied, at the same time especially for the mechanism for argument extraction, a relatively high accuracy is achieved in comparison to other studies that were taken into consideration for the completion of this Thesis. At the same, the process by which we attempted to extract suggestions to a great enough extent constitutes our own idea, given that in the past, few Systems that create that particular type of analysis had been implemented. Subsequently, up to this point, the results we received from the evaluation are very satisfying.\\
\\
For the completion of this Thesis, many arguments were dealt with, which were mainly algorithmic difficulties. A lot of studying was needed to be done in the available Related Work so that we can move on to the implementation. One more big difficulty was the incomplete knowledge we had about Machine Learning. All the difficulties were dealt with by studying the available bibliography.\\
\\
The first thought about Future work based on this Thesis is that further separation of Arguments could be done, as it was analysed in chapter 2. That means seeking to separate the part of the sentence in which the writer of the text claims something from the part that constitutes evaluation-validation. In the Future, we could also attempt Optimization of the suggestion extraction process, by trying other techniques. Finally, when it comes to the mechanism that extracts a writer's general knowledge on a matter, it would be important enough to test its efficiency on a better dataset.