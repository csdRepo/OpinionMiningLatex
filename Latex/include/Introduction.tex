% CREATED BY DAVID FRISK, 2015
\chapter{Introduction}
The goal of this Thesis is the study and creation of a system which will be able to derive and analyse information from citizens' commentaries on Bills which are available for public consultation from the Greek Government via the OpenGov website (\url{http://www.opengov.gr/}). More precisely, an effort is made to analyse information that exists mainly in unstructured writing by which someone can express their opinion on certain law articles of Bills about which the Greek Government has asked for the opinion of the Greek citizens, as well as their separation in suggestions that contain certain information. Also, we attempt to analyse whether the commentaries express a negative or a positive view on the laws or the Bill.\\
\\
Finally, information about whether a sentence supports the writer's opinion, i.e. whether this sentence constitutes a writer's argument as well as whether the writer suggests something else in the text's sentences. Based on the above, it is considered necessary to set apart three main and autonomous techniques which help in the extraction of this information. The first one is ``Argument Extraction'', which aims at finding the sentences with which the writer argues for or against a law. The second is the ``Extraction of  Suggestions and Counter-Suggestions'', which aims at finding the sentences in the text by which the writer makes a suggestion or a counter suggestion. The last one is ``General Opinion Extraction'', i.e. finding out whether the commentator of the law is positive or negative towards it.

\section{Motivation}
In the past years, a rapid increase in the information published on the World Wide Web has been noted. It is a fact that recent researches show that the rates keeps increasing very quickly. The phrase that is used in all scientific articles that study the increase of information on the Internet, \textit{``Internet is growing at lighting speed''} is characteristic.\\
\\
From the above ascertainment we can easily imagine that a big part of the information that is published does not constitute simply a part of a text in a random page. More precisely, it has been noted that the volume of information that is published in Social Media, Social Networks, Forums and generally in public consultation platforms keeps increasing at a rate proportional to  the increase of the total information in the World Wide Web.\\
\\
Therefore a list of questions comes up:\\

\begin{itemize}

	\item How can we extract data from unstructured plain texts?
	\item Is it possible to extract qualitative data (arguments) from these texts?
	\item Is it possible for the above to be done on every type of texts, without being affected by the subject and the size?\\

\end{itemize}

We can think of many more questions.\\
\\
Based on the above indicative questions, it makes sense to assume that it is necessary for such systems to exist/be improved/be perfectioned. In the research community, the interest for the creation of these  systems-tools has existed for years. Nevertheless, we have to  highlight that especially during the time that this Thesis is being written, many studies have been conducted and many systems have been created, which help a lot to carry out new studies of this kind (they will be analysed in the next chapter).\\
\\
Apart from the need that was described in the three questions, it is important to define some practical applications for the issues that are being studied in this Thesis.\\
\\
But first, we have to give a very brief description of the abilities of the system that was created at the same time with the writing of this Thesis (a brief one because they will be described in detail later on).\\
\\
First, we have to note that the texts we study are users' answers in the Greek Government's Public Consultation service.\\
\\
Functions:\\

\begin{itemize}

	\item Automatic extraction of proposals and counter proposal that users make in a text.
	\item Extraction of the argumentation that the writers compose.
	\item Extraction of the general agreement/disagreement of the writer, imprinted in one word, ``Positive'' or ``Negative''.\\

\end{itemize}

The above functions, in essence, could help many practical applications, not only in the field of Computer Science by adding the above functions in more complex systems but also in a wide variety of other fields.\\

\begin{itemize}

	\item \textbf{Search Engines:} So far, many search engines which have the ability to retrieve information that the user is looking for, exist. This search sometimes is done by using methods and algorithms from the field of Information Retrieval (TF-IDF, PageRank etc.) and some cases by using a Query Language if we refer to Link Open Data. In both cases, the way information is retrieved cannot offer more qualitative characteristics, especially if we think about the subject of argumentation. Therefore, let's imagine the abilities that a search engine, which could retrieve information and combine them with key words, would  provide us with. In order to make it clearer, we are going to give some examples in query form. \textit{``Find the arguments that are in favour of buying X car''}. \textit{``Find the arguments that contradict the arguments that are in favour of buying X car''}. \textit{``Find all the arguments that are against Bill Y''}, \textit{``''Find everyone who is against Bill Z''} etc. (The thought for the existence of such a search Engine has been expressed by Giorgos Flouris, Antonis Bikakis, Theodore Patkos, Dimitris Plexousakis. Globally Interconnecting Persuasive Arguments: The vision of the Persuasive Web. Technical Report. FORTH-ICS/TR-438, November 2013).
	\item \textbf{Journalism:} The creation of a system which could extract all the above information, would help this field a lot. A journalist could study the argumentation of politicians and draw conclusions about the opinions someone has expressed  in a specific amount of time. Of course, we could look for arguments and counterarguments on a matter and also we could check if certain individuals change their stance on a matter as time goes by. We can think of many other scenarios/possible applications.
	\item \textbf{Politics/Administration:} This tool could frame an online consultation service, especially in administration and decision making with the citizens' participation. Some of the functions that the system we develop offers could help in that direction. For example, we could retrieve all the counter proposals/disagreements.\\

\end{itemize}

\section{Argument Extraction}
First of all, in order to understand this technique, it is necessary to define the term ``Argument'', as well as the parts that constitute it.\\
\\
\textbf{Argument:} A reasoning with which someone supports or opposes a specific point of view.\\
\\
The parts of an Argument are ``Premise''-``Claim''-``Conclusion''. The premise is the sentence with which someone claims/assumes something. Claim is the sentence with which someone tries to reaffirm his assumption. Finally, the conclusion is considered to be the logical implication of the assumption and of the claim. Together with the parts of the argument, it is essential to define the connections between them.\cite{(6)}\cite{(7)}\\
\begin{itemize}

	\item A premise can be a claim and vice versa but not necessarily.
	\item A premise can oppose a claim or support it.\\

\end{itemize}

\begin{example}
``\textbf{(1)} Museums and Art galleries offer a better understanding of the Arts to the public, compared to the Internet. \textbf{(2)} Because in most museums and Art galleries, a detailed description of the artist's background is provided.''\\
\end{example}

In the above example (1.2.1), we can see an argument that is made up of the assumption \textbf{(2)} which is supported by the claim \textbf{(1)}.\\
\\
So, Argument Extraction is defined as the process of identifying arguments together with their components in the text. It is noted that it constitutes a complicated enough process, as the definition of a sentence as an argument with Grammar rules is not clear enough. This is because beyond Grammar rules, the understanding of the text's topic is necessary, as well as the semantics of the words that are directly linked to the topic. There are words whose interpretation changes according to the topic and the style of the text. For example, it is not possible to understand with grammatical analysis (no matter what depths it reaches), whether someone is being ironic or whether a phrase is used in a literal or a figurative sense, let alone in comments which are basically unstructured, brief and with many errors that many times, even if they are made under a specific topic, do not seem entirely relevant.\\
\\
In this Thesis we will focus on defining a sentence as ``argumentative'' or ``non argumentative''. A sentence is considered ``argumentative'' when it has one of the parts that were mentioned previously. In the previous examples, sentence \textbf{(1)} and sentence \textbf{(2)} are considered ``argumentative'' as sentence \textbf{(1)} was considered as ``claim'' and sentence \textbf{(2)} as ``conclusion''.

\section{Suggestion Extraction}
In this part,the clarification of the term ''suggestion'' is considered necessary.\\
\\
\textbf{Suggestion:} It is the reasoning with which someone suggests or makes a counter suggestion about a specific topic.\\
\\
Defining a sentence as ``suggestion'' is a clearer Grammatical process because there are specific verb moods that are used to suggest something, for example, the Subjunctive (``to do'', ``to go'') and the  Imperative (``do'', ``eat'') but also phrases like ``I suggest  to...'', ``It would be nice to...''.\\
\begin{example}
	``I suggest lowering the VAT at 10\% for all domestic products.''\\
\end{example}
\begin{example}
	``Lower the VAT at 10\% for all domestic products.''\\
\end{example}

In the example (2) the sentence constitutes  ``suggestion''. This is easily distinguished because the verb is in the Subjunctive and there's the expression ``I suggest..''. The example (3) also constitutes a suggestion because the verb is in the Imperative. It is noted that, as it is concluded from this project, despite the fact that to define a sentence as a ``suggestion'', Grammar is usually enough, there are cases when someone might come across sentences that meet the Grammar rules we set as prerequisite but they do not constitute  ``suggestions''. There are cases, for example, where a sentence that meets the criteria does not constitute a suggestion but the next sentence, or a close one, might be a suggestion.\\
\begin{example}
	``The changes that must be made are: Vat reduction at 10\%. Increase in checks for tax evasion.''\\
\end{example}
In the example (4), the sentence that follows the grammatical patterns for suggestions is the first one but, nevertheless, it is not characterised as a ``suggestion''. The next ones, though, are suggestions despite the fact that they do not follow the grammatical patterns for suggestions.





\section{Overall Opinion Extraction}
By the term  ``Overall Opinion Extraction'', we mean the effort to extract information about the way the writer of the comment expressed  his opinion, i.e. whether he is negative or positive towards the topic discussed. This extraction is a difficult task because there are sentences where someone can express a sentiment in an interpretatively controversial way.\\
\begin{example}
	``The weather was nice until it started raining and we didn't go fishing.''\\
\end{example}
\begin{example}
	``The weather was nice until it started raining and unfortunately we didn't go fishing.''\\
\end{example}
In the example (5), it is not clear whether the writer wrote that sentence with a negative or neutral sentiment, so any interpretation of the way he wrote it is not clear. In contrast, in the (6) example, it is more apparent that the writer has written the sentence to show his displeasure because of the existence of the word ``unfortunately''.