% CREATED BY DAVID FRISK, 2015
\begin{thebibliography}{69}

\bibitem{Reference} Wandhofer, T., Van Eeckhaute, C., Taylor, S., and Fernandez, M. (2012). Wegov analysis tools to connect policy makers with citizens online. In Proceedings of the tGov Conference May 2012, Brunel University.

\bibitem{Reference} Diakopoulos, N. A. and Shamma, D. A. (2010). Characterizing debate performance via aggregated twitter sentiment. In Proceedings of the SIGCHI Conference on Human Factors in Computing Systems.

\bibitem{Reference} Tumasjan, A., Sprenger, T. O., Sandner, P. G., and Welpe, I. M. (2010). Predicting elections with twitter: What 140 characters reveal about political sentiment.

\bibitem{Reference} Argument Extraction from News, Blogs, and Social Media Theodosis Goudas Department of Digital Systems, University of Piraeus, Christos Louizos, Department of Informatics \& Telecommunications University of Athens, Georgios Petasis  and Vangelis Karkaletsis Software and Knowledge Engineering Laboratory.

\bibitem{Reference} Simone Teufel. 1999. Argumentative Zoning: Information Extraction from Scientific Text. Ph.D. thesis, University of Edinburgh.

\bibitem{Reference} Alan Sergeant. 2013. Automatic argumentation extraction. In Proceedings of the 10th European Semantic Web Conference, ESWC ’13, pages 656–660,Montpellier, France.

\bibitem{Reference} Annotating Argument Components and Relations in Persuasive Essays C Stab,I Gurevych COLING 2014.


\end{thebibliography}
