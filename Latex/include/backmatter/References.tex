% CREATED BY DAVID FRISK, 2015
\begin{thebibliography}{69}

\bibitem {Refernece} Giorgos Flouris, Antonis Bikakis, Theodore Patkos, Dimitris Plexousakis. Globally Interconnecting Persuasive Arguments: The vision of the Persuasive Web. Technical Report. FORTH-ICS/TR-438, November 2013.

\bibitem{Reference} Wandhofer, T., Van Eeckhaute, C., Taylor, S., and Fernandez, M.  Wegov analysis tools to connect policy makers with citizens online. Brunel University. In Proceedings of the tGov Conference May 2012.

\bibitem{Reference} Diakopoulos, N. A. and Shamma, D. A. . Characterizing debate performance via aggregated twitter sentiment. In Proceedings of the SIGCHI Conference on Human Factors in Computing Systems. 2010.

\bibitem{Reference} Tumasjan, A., Sprenger, T. O., Sandner, P. G., and Welpe.Predicting elections with twitter: What 140 characters reveal about political sentiment. 2010.

\bibitem{Reference} Theodosis Goudas Department of Digital Systems, University of Piraeus, Christos Louizos. Argument Extraction from News, Blogs, and Social Media  Department of Informatics \& Telecommunications University of Athens, Georgios Petasis  and Vangelis Karkaletsis Software and Knowledge Engineering Laboratory.

\bibitem{Reference} Simone Teufel.  Argumentative Zoning: Information Extraction from Scientific Text. Ph.D. thesis, University of Edinburgh. 1999.

\bibitem{Reference} Alan Sergeant.  Automatic argumentation extraction. In Proceedings of the 10th European Semantic Web Conference, pages 656–660. ESWC 2013.

\bibitem{Reference} Annotating Argument Components and Relations in Persuasive Essays C Stab,I Gurevych COLING 2014.


\end{thebibliography}
