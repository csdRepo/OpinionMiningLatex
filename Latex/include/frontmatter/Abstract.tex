% CREATED BY DAVID FRISK, 2015
%An Informative Headline describing the Content of the Report\\
%A Subtitle that can be Very Much Longer if Necessary\\
Moschonas Giannis, Smyrnaios Giorgos\\
Computer Science Department\\
University of Crete \setlength{\parskip}{0.5cm}

\thispagestyle{plain}			% Supress header 
\setlength{\parskip}{0pt plus 1.0pt}
\section*{Abstract}
Opinion Mining is the field of Computer Science that studies the association between Natural Language and Computers in an attempt to mine semantically rich information from writing. More precisely, in the wider field of ``Natural Language Processing (NLP)'', ``Machine Translation'', ``Named Entity Recognition and Disambiguation'', ``Sentiment Analysis'' etc. are included. In this Thesis, a better approach to information extraction from plain texts is attempted. The texts contain commentaries of citizens on consultation of Laws issued by the Greek Government. Extraction of proposals and counter-proposals that the writers of these texts make, as well as of the arguments they formulate, is attempted. Finally, the general opinion of the writer is extracted, which is summarised by a ``Positive'' or ``Negative'' characterisation, depending on  the writer's view, which comes of the whole text. Mining of the above data is done entirely by analysing the texts through a three-stage analysis (which will be explained in detail in the following chapters of this Thesis) and by utilising techniques that are included in the wider spectrum of NLP (Information Retrieval, Part-of-Speech Tagging, Sentiment Analysis etc.). The results show that we can create realistic methods that can extract this type of information. This matter is important for the Scientific community because these methods can be put into use in a variety of other fields, apart from the field of Computer Science (e.g. Journalism, Politics etc.).





% KEYWORDS (MAXIMUM 10 WORDS)
\vfill
\textbf{Keywords:} argument extraction, sentiment, machine learning, suggestion extraction, POS Tagging, opinion mining, natural language processing.
